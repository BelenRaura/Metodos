% Options for packages loaded elsewhere
\PassOptionsToPackage{unicode}{hyperref}
\PassOptionsToPackage{hyphens}{url}
\PassOptionsToPackage{dvipsnames,svgnames,x11names}{xcolor}
%
\documentclass[
  letterpaper,
  DIV=11,
  numbers=noendperiod]{scrartcl}

\usepackage{amsmath,amssymb}
\usepackage{iftex}
\ifPDFTeX
  \usepackage[T1]{fontenc}
  \usepackage[utf8]{inputenc}
  \usepackage{textcomp} % provide euro and other symbols
\else % if luatex or xetex
  \usepackage{unicode-math}
  \defaultfontfeatures{Scale=MatchLowercase}
  \defaultfontfeatures[\rmfamily]{Ligatures=TeX,Scale=1}
\fi
\usepackage{lmodern}
\ifPDFTeX\else  
    % xetex/luatex font selection
\fi
% Use upquote if available, for straight quotes in verbatim environments
\IfFileExists{upquote.sty}{\usepackage{upquote}}{}
\IfFileExists{microtype.sty}{% use microtype if available
  \usepackage[]{microtype}
  \UseMicrotypeSet[protrusion]{basicmath} % disable protrusion for tt fonts
}{}
\makeatletter
\@ifundefined{KOMAClassName}{% if non-KOMA class
  \IfFileExists{parskip.sty}{%
    \usepackage{parskip}
  }{% else
    \setlength{\parindent}{0pt}
    \setlength{\parskip}{6pt plus 2pt minus 1pt}}
}{% if KOMA class
  \KOMAoptions{parskip=half}}
\makeatother
\usepackage{xcolor}
\setlength{\emergencystretch}{3em} % prevent overfull lines
\setcounter{secnumdepth}{-\maxdimen} % remove section numbering
% Make \paragraph and \subparagraph free-standing
\makeatletter
\ifx\paragraph\undefined\else
  \let\oldparagraph\paragraph
  \renewcommand{\paragraph}{
    \@ifstar
      \xxxParagraphStar
      \xxxParagraphNoStar
  }
  \newcommand{\xxxParagraphStar}[1]{\oldparagraph*{#1}\mbox{}}
  \newcommand{\xxxParagraphNoStar}[1]{\oldparagraph{#1}\mbox{}}
\fi
\ifx\subparagraph\undefined\else
  \let\oldsubparagraph\subparagraph
  \renewcommand{\subparagraph}{
    \@ifstar
      \xxxSubParagraphStar
      \xxxSubParagraphNoStar
  }
  \newcommand{\xxxSubParagraphStar}[1]{\oldsubparagraph*{#1}\mbox{}}
  \newcommand{\xxxSubParagraphNoStar}[1]{\oldsubparagraph{#1}\mbox{}}
\fi
\makeatother

\usepackage{color}
\usepackage{fancyvrb}
\newcommand{\VerbBar}{|}
\newcommand{\VERB}{\Verb[commandchars=\\\{\}]}
\DefineVerbatimEnvironment{Highlighting}{Verbatim}{commandchars=\\\{\}}
% Add ',fontsize=\small' for more characters per line
\usepackage{framed}
\definecolor{shadecolor}{RGB}{241,243,245}
\newenvironment{Shaded}{\begin{snugshade}}{\end{snugshade}}
\newcommand{\AlertTok}[1]{\textcolor[rgb]{0.68,0.00,0.00}{#1}}
\newcommand{\AnnotationTok}[1]{\textcolor[rgb]{0.37,0.37,0.37}{#1}}
\newcommand{\AttributeTok}[1]{\textcolor[rgb]{0.40,0.45,0.13}{#1}}
\newcommand{\BaseNTok}[1]{\textcolor[rgb]{0.68,0.00,0.00}{#1}}
\newcommand{\BuiltInTok}[1]{\textcolor[rgb]{0.00,0.23,0.31}{#1}}
\newcommand{\CharTok}[1]{\textcolor[rgb]{0.13,0.47,0.30}{#1}}
\newcommand{\CommentTok}[1]{\textcolor[rgb]{0.37,0.37,0.37}{#1}}
\newcommand{\CommentVarTok}[1]{\textcolor[rgb]{0.37,0.37,0.37}{\textit{#1}}}
\newcommand{\ConstantTok}[1]{\textcolor[rgb]{0.56,0.35,0.01}{#1}}
\newcommand{\ControlFlowTok}[1]{\textcolor[rgb]{0.00,0.23,0.31}{\textbf{#1}}}
\newcommand{\DataTypeTok}[1]{\textcolor[rgb]{0.68,0.00,0.00}{#1}}
\newcommand{\DecValTok}[1]{\textcolor[rgb]{0.68,0.00,0.00}{#1}}
\newcommand{\DocumentationTok}[1]{\textcolor[rgb]{0.37,0.37,0.37}{\textit{#1}}}
\newcommand{\ErrorTok}[1]{\textcolor[rgb]{0.68,0.00,0.00}{#1}}
\newcommand{\ExtensionTok}[1]{\textcolor[rgb]{0.00,0.23,0.31}{#1}}
\newcommand{\FloatTok}[1]{\textcolor[rgb]{0.68,0.00,0.00}{#1}}
\newcommand{\FunctionTok}[1]{\textcolor[rgb]{0.28,0.35,0.67}{#1}}
\newcommand{\ImportTok}[1]{\textcolor[rgb]{0.00,0.46,0.62}{#1}}
\newcommand{\InformationTok}[1]{\textcolor[rgb]{0.37,0.37,0.37}{#1}}
\newcommand{\KeywordTok}[1]{\textcolor[rgb]{0.00,0.23,0.31}{\textbf{#1}}}
\newcommand{\NormalTok}[1]{\textcolor[rgb]{0.00,0.23,0.31}{#1}}
\newcommand{\OperatorTok}[1]{\textcolor[rgb]{0.37,0.37,0.37}{#1}}
\newcommand{\OtherTok}[1]{\textcolor[rgb]{0.00,0.23,0.31}{#1}}
\newcommand{\PreprocessorTok}[1]{\textcolor[rgb]{0.68,0.00,0.00}{#1}}
\newcommand{\RegionMarkerTok}[1]{\textcolor[rgb]{0.00,0.23,0.31}{#1}}
\newcommand{\SpecialCharTok}[1]{\textcolor[rgb]{0.37,0.37,0.37}{#1}}
\newcommand{\SpecialStringTok}[1]{\textcolor[rgb]{0.13,0.47,0.30}{#1}}
\newcommand{\StringTok}[1]{\textcolor[rgb]{0.13,0.47,0.30}{#1}}
\newcommand{\VariableTok}[1]{\textcolor[rgb]{0.07,0.07,0.07}{#1}}
\newcommand{\VerbatimStringTok}[1]{\textcolor[rgb]{0.13,0.47,0.30}{#1}}
\newcommand{\WarningTok}[1]{\textcolor[rgb]{0.37,0.37,0.37}{\textit{#1}}}

\providecommand{\tightlist}{%
  \setlength{\itemsep}{0pt}\setlength{\parskip}{0pt}}\usepackage{longtable,booktabs,array}
\usepackage{calc} % for calculating minipage widths
% Correct order of tables after \paragraph or \subparagraph
\usepackage{etoolbox}
\makeatletter
\patchcmd\longtable{\par}{\if@noskipsec\mbox{}\fi\par}{}{}
\makeatother
% Allow footnotes in longtable head/foot
\IfFileExists{footnotehyper.sty}{\usepackage{footnotehyper}}{\usepackage{footnote}}
\makesavenoteenv{longtable}
\usepackage{graphicx}
\makeatletter
\def\maxwidth{\ifdim\Gin@nat@width>\linewidth\linewidth\else\Gin@nat@width\fi}
\def\maxheight{\ifdim\Gin@nat@height>\textheight\textheight\else\Gin@nat@height\fi}
\makeatother
% Scale images if necessary, so that they will not overflow the page
% margins by default, and it is still possible to overwrite the defaults
% using explicit options in \includegraphics[width, height, ...]{}
\setkeys{Gin}{width=\maxwidth,height=\maxheight,keepaspectratio}
% Set default figure placement to htbp
\makeatletter
\def\fps@figure{htbp}
\makeatother

\KOMAoption{captions}{tableheading}
\makeatletter
\@ifpackageloaded{caption}{}{\usepackage{caption}}
\AtBeginDocument{%
\ifdefined\contentsname
  \renewcommand*\contentsname{Table of contents}
\else
  \newcommand\contentsname{Table of contents}
\fi
\ifdefined\listfigurename
  \renewcommand*\listfigurename{List of Figures}
\else
  \newcommand\listfigurename{List of Figures}
\fi
\ifdefined\listtablename
  \renewcommand*\listtablename{List of Tables}
\else
  \newcommand\listtablename{List of Tables}
\fi
\ifdefined\figurename
  \renewcommand*\figurename{Figure}
\else
  \newcommand\figurename{Figure}
\fi
\ifdefined\tablename
  \renewcommand*\tablename{Table}
\else
  \newcommand\tablename{Table}
\fi
}
\@ifpackageloaded{float}{}{\usepackage{float}}
\floatstyle{ruled}
\@ifundefined{c@chapter}{\newfloat{codelisting}{h}{lop}}{\newfloat{codelisting}{h}{lop}[chapter]}
\floatname{codelisting}{Listing}
\newcommand*\listoflistings{\listof{codelisting}{List of Listings}}
\makeatother
\makeatletter
\makeatother
\makeatletter
\@ifpackageloaded{caption}{}{\usepackage{caption}}
\@ifpackageloaded{subcaption}{}{\usepackage{subcaption}}
\makeatother

\ifLuaTeX
  \usepackage{selnolig}  % disable illegal ligatures
\fi
\usepackage{bookmark}

\IfFileExists{xurl.sty}{\usepackage{xurl}}{} % add URL line breaks if available
\urlstyle{same} % disable monospaced font for URLs
\hypersetup{
  pdftitle={Deber 2},
  colorlinks=true,
  linkcolor={blue},
  filecolor={Maroon},
  citecolor={Blue},
  urlcolor={Blue},
  pdfcreator={LaTeX via pandoc}}


\title{Deber 2}
\author{}
\date{}

\begin{document}
\maketitle


\section{CONJUNTO DE EJERCICIOS 1.3}\label{conjunto-de-ejercicios-1.3}

\subsection{1. Aritmética de corte de tres
dígitos}\label{aritmuxe9tica-de-corte-de-tres-duxedgitos}

Utilice aritmética de corte de tres dígitos para calcular las siguientes
sumas. Para cada parte, ¿qué método es más preciso y por qué?

\subsubsection{a.}\label{a.}

\[
\sum_{i=1}^{10} \frac{1}{i^2} 
\
     Primero por:   
\[
1 + \frac{1}{4} + \ldots + \frac{1}{100} 
\]
    y luego por:
\[
\frac{1}{100} + \frac{1}{81} + \ldots + 1
\\]

\begin{Shaded}
\begin{Highlighting}[]
\KeywordTok{def}\NormalTok{ truncar(valor: }\BuiltInTok{float}\NormalTok{, decimales: }\BuiltInTok{int}\NormalTok{) }\OperatorTok{{-}\textgreater{}} \BuiltInTok{float}\NormalTok{:}
\NormalTok{    factor }\OperatorTok{=} \DecValTok{10} \OperatorTok{**}\NormalTok{ decimales}
    \ControlFlowTok{return} \BuiltInTok{int}\NormalTok{(valor }\OperatorTok{*}\NormalTok{ factor) }\OperatorTok{/}\NormalTok{ factor}

\KeywordTok{def}\NormalTok{ suma\_Uno\_mas(n: }\BuiltInTok{int}\NormalTok{) }\OperatorTok{{-}\textgreater{}} \BuiltInTok{float}\NormalTok{:}
\NormalTok{    primer }\OperatorTok{=} \FloatTok{0.0}  \CommentTok{\# Inicializar la suma en 0}
    \ControlFlowTok{for}\NormalTok{ i }\KeywordTok{in} \BuiltInTok{range}\NormalTok{(}\DecValTok{1}\NormalTok{, n }\OperatorTok{+} \DecValTok{1}\NormalTok{):  }\CommentTok{\# Iterar desde 1 hasta n}
\NormalTok{        term }\OperatorTok{=}\NormalTok{ truncar(}\DecValTok{1} \OperatorTok{/}\NormalTok{ (i }\OperatorTok{**} \DecValTok{2}\NormalTok{), }\DecValTok{3}\NormalTok{)  }\CommentTok{\# Truncar 1/i\^{}2 a 3 cifras decimales}
\NormalTok{        primer }\OperatorTok{+=}\NormalTok{ term  }\CommentTok{\# Sumar el término truncado}
    \ControlFlowTok{return}\NormalTok{ primer  }\CommentTok{\# Devolver la suma}

\NormalTok{a }\OperatorTok{=}\NormalTok{ suma\_Uno\_mas(}\DecValTok{10}\NormalTok{)  }
\BuiltInTok{print}\NormalTok{(a)  }\CommentTok{\# Mostrar el resultado}
\end{Highlighting}
\end{Shaded}

\begin{verbatim}
1.547
\end{verbatim}

\begin{Shaded}
\begin{Highlighting}[]
\KeywordTok{def}\NormalTok{ suma\_al\_reves(n: }\BuiltInTok{int}\NormalTok{) }\OperatorTok{{-}\textgreater{}} \BuiltInTok{float}\NormalTok{:}
\NormalTok{    suma }\OperatorTok{=} \FloatTok{0.0}  \CommentTok{\# Inicializar la suma en 0}
    \ControlFlowTok{for}\NormalTok{ i }\KeywordTok{in} \BuiltInTok{range}\NormalTok{(n, }\DecValTok{0}\NormalTok{, }\OperatorTok{{-}}\DecValTok{1}\NormalTok{):  }\CommentTok{\# Desde n hasta 1}
\NormalTok{        term }\OperatorTok{=}\NormalTok{ truncar(}\DecValTok{1} \OperatorTok{/}\NormalTok{ (i }\OperatorTok{**} \DecValTok{2}\NormalTok{), }\DecValTok{3}\NormalTok{)  }\CommentTok{\# Truncar 1/i\^{}2 a 3 cifras decimales}
\NormalTok{        suma }\OperatorTok{+=}\NormalTok{ term  }\CommentTok{\# Sumar el término truncado}
    \ControlFlowTok{return}\NormalTok{ suma  }\CommentTok{\# Devolver la suma}

\NormalTok{resultado }\OperatorTok{=}\NormalTok{ suma\_al\_reves(}\DecValTok{10}\NormalTok{)}
\BuiltInTok{print}\NormalTok{(resultado)  }\CommentTok{\# Mostrar el resultado}
\end{Highlighting}
\end{Shaded}

\begin{verbatim}
1.547
\end{verbatim}

\subsubsection{b.}\label{b.}

\[\
\sum_{i=1}^{10} \frac{1}{i^3}
\]
Primero por:
\[
1 + \frac{1}{8} + \frac{1}{27} + \ldots + \frac{1}{1000}
\]
y luego por:
\[
\frac{1}{1000} + \frac{1}{729} + \ldots + 1
\\]

\begin{Shaded}
\begin{Highlighting}[]


\KeywordTok{def}\NormalTok{ suma\_Cubica(n: }\BuiltInTok{int}\NormalTok{) }\OperatorTok{{-}\textgreater{}} \BuiltInTok{float}\NormalTok{:}
\NormalTok{    primer }\OperatorTok{=} \FloatTok{0.0}  \CommentTok{\# Inicializar la suma en 0}
    \ControlFlowTok{for}\NormalTok{ i }\KeywordTok{in} \BuiltInTok{range}\NormalTok{(}\DecValTok{1}\NormalTok{, n }\OperatorTok{+} \DecValTok{1}\NormalTok{):  }\CommentTok{\# Iterar desde 1 hasta n}
\NormalTok{        term }\OperatorTok{=}\NormalTok{ truncar(}\DecValTok{1} \OperatorTok{/}\NormalTok{ (i }\OperatorTok{**} \DecValTok{3}\NormalTok{), }\DecValTok{3}\NormalTok{)  }\CommentTok{\# Truncar 1/i\^{}2 a 3 cifras decimales}
\NormalTok{        primer }\OperatorTok{+=}\NormalTok{ term  }\CommentTok{\# Sumar el término truncado}
    \ControlFlowTok{return}\NormalTok{ primer  }\CommentTok{\# Devolver la suma}

\NormalTok{a }\OperatorTok{=}\NormalTok{ suma\_Cubica(}\DecValTok{10}\NormalTok{)  }
\BuiltInTok{print}\NormalTok{(a)}
\KeywordTok{def}\NormalTok{ suma\_al\_reves(n: }\BuiltInTok{int}\NormalTok{) }\OperatorTok{{-}\textgreater{}} \BuiltInTok{float}\NormalTok{:}
\NormalTok{    suma }\OperatorTok{=} \FloatTok{0.0}  \CommentTok{\# Inicializar la suma en 0}
    \ControlFlowTok{for}\NormalTok{ i }\KeywordTok{in} \BuiltInTok{range}\NormalTok{(n, }\DecValTok{0}\NormalTok{, }\OperatorTok{{-}}\DecValTok{1}\NormalTok{):  }\CommentTok{\# Desde n hasta 1}
\NormalTok{        term }\OperatorTok{=}\NormalTok{ truncar(}\DecValTok{1} \OperatorTok{/}\NormalTok{ (i }\OperatorTok{**} \DecValTok{3}\NormalTok{), }\DecValTok{3}\NormalTok{)  }\CommentTok{\# Truncar 1/i\^{}3 a 3 cifras decimales}
\NormalTok{        suma }\OperatorTok{+=}\NormalTok{ term  }\CommentTok{\# Sumar el término truncado}
    \ControlFlowTok{return}\NormalTok{ suma  }\CommentTok{\# Devolver la suma}

\NormalTok{resultado }\OperatorTok{=}\NormalTok{ suma\_al\_reves(}\DecValTok{10}\NormalTok{)}
\BuiltInTok{print}\NormalTok{(resultado)  }\CommentTok{\# Mostrar el resultado}

\end{Highlighting}
\end{Shaded}

\begin{verbatim}
1.1939999999999995
1.194
\end{verbatim}

El método más preciso es el de la suma directa. Esto se debe a que:

La suma directa mantiene la precisión al sumar términos más grandes
primero, lo que minimiza la pérdida de significancia en las cifras
cuando se suman términos más pequeños. En la suma inversa, los errores
de redondeo son más propensos a afectar el resultado final debido a la
naturaleza de los números involucrados.

\subsection{2. Serie de Maclaurin para la función
arcotangente}\label{serie-de-maclaurin-para-la-funciuxf3n-arcotangente}

La serie de Maclaurin para la función arcotangente converge para
\[\-1 < x \leq 1\\] y está dada por:

\[
\arctan x = \lim_{n \to \infty} P_n(x) = \lim_{n \to \infty} \sum_{i=1}^{n} (-1)^{i+1} \frac{x^{2i-1}}{2i-1}
\]

\subsubsection{a.}\label{a.-1}

Utilice el hecho de que \[\tan \frac{\pi}{4} = 1\ \] para determinar el
número (n) de términos de la serie que se necesita sumar para garantizar
que \[\|4P_n(1) - \pi| < 10^{-3}\\].

\subsection{Pasos para resolver el
problema:}\label{pasos-para-resolver-el-problema}

\begin{enumerate}
\def\labelenumi{\arabic{enumi}.}
\item
  \textbf{Definir la serie de Maclaurin}

  para \[ arctan(1) \] :

  Para ( x = 1 ), la serie se convierte en:

  \[ P_n(1) = \sum_{i=1}^n (-1)^{i+1} \frac{1^{2i-1}}{2i-1} = \sum_{i=1}^n (-1)^{i+1} \frac{1}{2i-1}
   \]

  Esto se debe a que \[\ 1^{2i-1} = 1 \\] para cualquier i .
\item
  \textbf{Condición de precisión}:

  La serie \[ P_n(1) \] se multiplica por 4 para aproximar π. Entonces,
  encontrar un n tal que:
\end{enumerate}

\[
   \left| 4 P_n(1) - \pi \right| < 10^{-3}
\]

\begin{enumerate}
\def\labelenumi{\arabic{enumi}.}
\setcounter{enumi}{4}
\item
  \textbf{Cálculo de ( n )}:

  Utilizaremos un bucle para calcular la suma hasta que la diferencia
  con π sea menor a \[\ 10^{-3} \].
\end{enumerate}

\begin{Shaded}
\begin{Highlighting}[]

\ImportTok{import}\NormalTok{ math}

\KeywordTok{def}\NormalTok{ calcular\_n(precision: }\BuiltInTok{float}\NormalTok{) }\OperatorTok{{-}\textgreater{}} \BuiltInTok{int}\NormalTok{:}
\NormalTok{    pi\_approx }\OperatorTok{=} \FloatTok{0.0}
\NormalTok{    n }\OperatorTok{=} \DecValTok{0}
\NormalTok{    x }\OperatorTok{=} \DecValTok{1}  \CommentTok{\# Definiendo el valor de x según el problema}
    \ControlFlowTok{while} \VariableTok{True}\NormalTok{:}
\NormalTok{        n }\OperatorTok{+=} \DecValTok{1}
\NormalTok{        term }\OperatorTok{=}\NormalTok{ ((}\OperatorTok{{-}}\DecValTok{1}\NormalTok{)}\OperatorTok{**}\NormalTok{(n }\OperatorTok{+} \DecValTok{1}\NormalTok{)) }\OperatorTok{*}\NormalTok{ (x}\OperatorTok{**}\NormalTok{(}\DecValTok{2} \OperatorTok{*}\NormalTok{ n }\OperatorTok{{-}} \DecValTok{1}\NormalTok{) }\OperatorTok{/}\NormalTok{ (}\DecValTok{2} \OperatorTok{*}\NormalTok{ n }\OperatorTok{{-}} \DecValTok{1}\NormalTok{))  }\CommentTok{\# Término de la serie}
\NormalTok{        pi\_approx }\OperatorTok{+=}\NormalTok{ term}
        \ControlFlowTok{if} \BuiltInTok{abs}\NormalTok{(}\DecValTok{4} \OperatorTok{*}\NormalTok{ pi\_approx }\OperatorTok{{-}}\NormalTok{ math.pi) }\OperatorTok{\textless{}}\NormalTok{ precision:}
            \ControlFlowTok{break}
    \ControlFlowTok{return}\NormalTok{ n}

\NormalTok{n\_necesario }\OperatorTok{=}\NormalTok{ calcular\_n(}\DecValTok{10}\OperatorTok{**{-}}\DecValTok{3}\NormalTok{)}
\BuiltInTok{print}\NormalTok{(n\_necesario)}




\end{Highlighting}
\end{Shaded}

\begin{verbatim}
1000
\end{verbatim}

\subsubsection{b.}\label{b.-1}

El lenguaje de programación C++ requiere que el valor de π se encuentre
dentro de (10\^{}\{-10\}). ¿Cuántos términos de la serie se necesitarían
sumar para obtener este grado de precisión?

\begin{Shaded}
\begin{Highlighting}[]
\ImportTok{from}\NormalTok{ sympy }\ImportTok{import} \OperatorTok{*}

\CommentTok{\# Definimos la serie de Maclaurin}
\NormalTok{n }\OperatorTok{=}\NormalTok{ symbols(}\StringTok{\textquotesingle{}n\textquotesingle{}}\NormalTok{)}
\NormalTok{term }\OperatorTok{=}\NormalTok{ (}\OperatorTok{{-}}\DecValTok{1}\NormalTok{)}\OperatorTok{**}\NormalTok{(n}\OperatorTok{+}\DecValTok{1}\NormalTok{) }\OperatorTok{/}\NormalTok{ (}\DecValTok{2}\OperatorTok{*}\NormalTok{n }\OperatorTok{{-}} \DecValTok{1}\NormalTok{)}
\NormalTok{partial\_sum }\OperatorTok{=}\NormalTok{ Sum(term, (n, }\DecValTok{1}\NormalTok{, n))}

\CommentTok{\# Función para calcular el error}
\KeywordTok{def}\NormalTok{ error(n):}
\NormalTok{    approx\_pi }\OperatorTok{=} \DecValTok{4} \OperatorTok{*}\NormalTok{ partial\_sum.subs(n, n).evalf()}
    \ControlFlowTok{return} \BuiltInTok{abs}\NormalTok{(approx\_pi }\OperatorTok{{-}}\NormalTok{ pi)}


\NormalTok{n }\OperatorTok{=} \DecValTok{1}
\ControlFlowTok{while}\NormalTok{ error(n) }\OperatorTok{\textgreater{}=} \FloatTok{1e{-}10}\NormalTok{:}
\NormalTok{    n }\OperatorTok{+=} \DecValTok{1}
\BuiltInTok{print}\NormalTok{(}\StringTok{"Para una precisión de 10\^{}{-}10, se necesitan al menos"}\NormalTok{, n, }\StringTok{"términos."}\NormalTok{)}
\end{Highlighting}
\end{Shaded}

\subsection{3. Otra fórmula para calcular
π}\label{otra-fuxf3rmula-para-calcular-ux3c0}

Otra fórmula para calcular π se puede deducir a partir de la identidad:
\[
\frac{\pi}{4} = 4 \text{arctan } \frac{1}{5} - \text{arctan } \frac{1}{239}
\\] Determine el número de términos que se deben sumar para garantizar
una aproximación de π dentro de (10\^{}\{-3\}).

\subsubsection{Resolución del ejercicio
3}\label{resoluciuxf3n-del-ejercicio-3}

\paragraph{Identidad dada:}\label{identidad-dada}

Se nos proporciona la siguiente identidad para calcular π:

\[
\frac{\pi}{4} = 4 \arctan\left(\frac{1}{5}\right) - \arctan\left(\frac{1}{239}\right)
\\]

Esta fórmula se conoce como una fórmula de Machin para aproximar el
valor de π. La idea es usar la serie de Maclaurin para (\arctan(x)) y
determinar cuántos términos debemos sumar para alcanzar una precisión de
(10\^{}\{-3\}).

\paragraph{\texorpdfstring{Serie de Maclaurin para
(\arctan(x)):}{Serie de Maclaurin para ((x)):}}\label{serie-de-maclaurin-para-x}

La serie de Maclaurin para la función (\arctan(x)) está dada por:

\[
\arctan(x) = \sum_{i=0}^n (-1)^i \frac{x^{2i+1}}{2i+1}
\\]

\paragraph{Expresión de π:}\label{expresiuxf3n-de-ux3c0}

Utilizando la identidad dada, podemos escribir la expresión para π como:

\[
\pi \approx 4 \left[ 4 \sum_{i=0}^n (-1)^i \frac{\left(\frac{1}{5}\right)^{2i+1}}{2i+1} - \sum_{i=0}^n (-1)^i \frac{\left(\frac{1}{239}\right)^{2i+1}}{2i+1} \right]
\\]

Nuestro objetivo es determinar cuántos términos ((n)) debemos sumar para
que el valor calculado de (\pi) esté dentro de una precisión de
(10\^{}\{-3\}), es decir, cumplir con la siguiente condición:

\[
\left|\pi_{\text{aprox}} - \pi\right| < 10^{-3}
\\]

\begin{Shaded}
\begin{Highlighting}[]
\ImportTok{import}\NormalTok{ math}

\KeywordTok{def}\NormalTok{ calcular\_n(precision: }\BuiltInTok{float}\NormalTok{) }\OperatorTok{{-}\textgreater{}} \BuiltInTok{int}\NormalTok{:}
\NormalTok{    pi\_approx }\OperatorTok{=} \FloatTok{0.0}
\NormalTok{    n }\OperatorTok{=} \DecValTok{0}
    \ControlFlowTok{while} \VariableTok{True}\NormalTok{:}
\NormalTok{        n }\OperatorTok{+=} \DecValTok{1}
\NormalTok{        term1 }\OperatorTok{=}\NormalTok{ ((}\OperatorTok{{-}}\DecValTok{1}\NormalTok{) }\OperatorTok{**}\NormalTok{ (n }\OperatorTok{+} \DecValTok{1}\NormalTok{)) }\OperatorTok{*}\NormalTok{ (}\DecValTok{1} \OperatorTok{/} \DecValTok{5}\NormalTok{) }\OperatorTok{**}\NormalTok{ (}\DecValTok{2} \OperatorTok{*}\NormalTok{ n }\OperatorTok{{-}} \DecValTok{1}\NormalTok{) }\OperatorTok{/}\NormalTok{ (}\DecValTok{2} \OperatorTok{*}\NormalTok{ n }\OperatorTok{{-}} \DecValTok{1}\NormalTok{)}
\NormalTok{        term2 }\OperatorTok{=}\NormalTok{ ((}\OperatorTok{{-}}\DecValTok{1}\NormalTok{) }\OperatorTok{**}\NormalTok{ (n }\OperatorTok{+} \DecValTok{1}\NormalTok{)) }\OperatorTok{*}\NormalTok{ (}\DecValTok{1} \OperatorTok{/} \DecValTok{239}\NormalTok{) }\OperatorTok{**}\NormalTok{ (}\DecValTok{2} \OperatorTok{*}\NormalTok{ n }\OperatorTok{{-}} \DecValTok{1}\NormalTok{) }\OperatorTok{/}\NormalTok{ (}\DecValTok{2} \OperatorTok{*}\NormalTok{ n }\OperatorTok{{-}} \DecValTok{1}\NormalTok{)}
        
        \CommentTok{\# Aproximación a pi usando la fórmula de Machin}
\NormalTok{        pi\_approx }\OperatorTok{=} \DecValTok{4} \OperatorTok{*}\NormalTok{ (}\DecValTok{4} \OperatorTok{*}\NormalTok{ term1 }\OperatorTok{{-}}\NormalTok{ term2)}
        
        \CommentTok{\# Comprobar si la aproximación de pi cumple con la precisión requerida}
        \ControlFlowTok{if} \BuiltInTok{abs}\NormalTok{(pi\_approx }\OperatorTok{{-}}\NormalTok{ math.pi) }\OperatorTok{\textless{}}\NormalTok{ precision:}
            \ControlFlowTok{break}
    \ControlFlowTok{return}\NormalTok{ n}

\CommentTok{\# Determinar el número de términos necesarios para una precisión de 10\^{}({-}3)}
\NormalTok{n\_necesario }\OperatorTok{=}\NormalTok{ calcular\_n(}\DecValTok{10}\OperatorTok{**{-}}\DecValTok{3}\NormalTok{)}
\BuiltInTok{print}\NormalTok{(}\SpecialStringTok{f"Número de términos necesarios: }\SpecialCharTok{\{}\NormalTok{n\_necesario}\SpecialCharTok{\}}\SpecialStringTok{"}\NormalTok{)}
\end{Highlighting}
\end{Shaded}

\subsection{5. Cálculos de suma}\label{cuxe1lculos-de-suma}

\subsubsection{a.}\label{a.-2}

¿Cuántas multiplicaciones y sumas se requieren para determinar una suma
de la forma \[\
\sum_{i=1}^{n} \sum_{j=1}^{n} a_i b_j?
\\]

\subsubsection{Resolución del ejercicio
5}\label{resoluciuxf3n-del-ejercicio-5}

Queremos determinar cuántas multiplicaciones y sumas se requieren para
calcular la expresión data;

\[
\sum_{i=1}^n \sum_{j=1}^i a_i b_j
\]

\paragraph{Análisis:}\label{anuxe1lisis}

\begin{itemize}
\tightlist
\item
  \textbf{Bucle externo}: Recorre el índice ( i ) desde ( 1 ) hasta ( n
  ).
\item
  \textbf{Bucle interno}: Recorre el índice ( j ) desde ( 1 ) hasta ( i
  ).
\end{itemize}

En cada iteración del bucle interno, se realiza una multiplicación (
a\_i \cdot b\_j ) y luego se suma el resultado al total acumulado.

\paragraph{Total de operaciones:}\label{total-de-operaciones}

\begin{enumerate}
\def\labelenumi{\arabic{enumi}.}
\item
  \textbf{Multiplicaciones}: Para cada ( i ), se realizan ( i )
  multiplicaciones, por lo que el total de multiplicaciones es:

  {[} 1 + 2 + 3 + \ldots + n = \frac{n(n+1)}{2} {]}
\item
  \textbf{Sumas}: En cada iteración del bucle interno se suma el
  resultado de la multiplicación al total acumulado, lo que implica (
  \frac{n(n+1)}{2} ) sumas.
\end{enumerate}

\paragraph{Conclusión:}\label{conclusiuxf3n}

Para determinar una suma de la forma (\sum\emph{\{i=1\}\^{}n
\sum}\{j=1\}\^{}i a\_i b\_j), se requieren:

\begin{itemize}
\tightlist
\item
  (\frac{n(n+1)}{2}) multiplicaciones.
\item
  (\frac{n(n+1)}{2}) sumas.
\end{itemize}

\begin{Shaded}
\begin{Highlighting}[]
\KeywordTok{def}\NormalTok{ contar\_operaciones(n):}
    \CommentTok{\# Cálculo del número de multiplicaciones y sumas}
\NormalTok{    num\_multiplicaciones }\OperatorTok{=}\NormalTok{ (n }\OperatorTok{*}\NormalTok{ (n }\OperatorTok{+} \DecValTok{1}\NormalTok{)) }\OperatorTok{//} \DecValTok{2}
\NormalTok{    num\_sumas }\OperatorTok{=}\NormalTok{ num\_multiplicaciones  }\CommentTok{\# La cantidad de sumas es igual a la cantidad de multiplicaciones}

    \ControlFlowTok{return}\NormalTok{ num\_multiplicaciones, num\_sumas}

\CommentTok{\# Ejemplo de uso}
\NormalTok{n }\OperatorTok{=} \DecValTok{5}
\NormalTok{multiplicaciones, sumas }\OperatorTok{=}\NormalTok{ contar\_operaciones(n)}
\BuiltInTok{print}\NormalTok{(}\SpecialStringTok{f"Para n = }\SpecialCharTok{\{}\NormalTok{n}\SpecialCharTok{\}}\SpecialStringTok{:"}\NormalTok{)}
\BuiltInTok{print}\NormalTok{(}\SpecialStringTok{f"Multiplicaciones necesarias: }\SpecialCharTok{\{}\NormalTok{multiplicaciones}\SpecialCharTok{\}}\SpecialStringTok{"}\NormalTok{)}
\BuiltInTok{print}\NormalTok{(}\SpecialStringTok{f"Sumas necesarias: }\SpecialCharTok{\{}\NormalTok{sumas}\SpecialCharTok{\}}\SpecialStringTok{"}\NormalTok{)}
\end{Highlighting}
\end{Shaded}

\begin{verbatim}
Para n = 5:
Multiplicaciones necesarias: 15
Sumas necesarias: 15
\end{verbatim}

\subsubsection{b.}\label{b.-2}

Modifique la suma en la parte a) a un formato equivalente que reduzca el
número de cálculos.

\subsubsection{Reorganización de la
suma}\label{reorganizaciuxf3n-de-la-suma}

En la \textbf{Parte b}, se pide modificar la suma para reducir el número
de cálculos. La suma original es de la forma:

\[
\sum_{i=1}^{n} \sum_{j=1}^{i} a_i b_j
\]

Podemos reorganizar esta suma de la siguiente manera:

\[
\sum_{i=1}^{n} \sum_{j=1}^{i} a_i b_j = \sum_{j=1}^{n} b_j \sum_{i=j}^{n} a_i
\]

\subsubsection{Justificación de la nueva
forma:}\label{justificaciuxf3n-de-la-nueva-forma}

En la suma original, para cada ( i ), recorremos desde ( j = 1 ) hasta (
j = i ). Esto implica realizar muchas multiplicaciones repetidas.

En la nueva forma, notamos que para un valor fijo de ( j ), el término (
b\_j ) permanece constante, mientras que ( a\_i ) se suma desde ( i = j
) hasta ( i = n ). Por lo tanto, reorganizamos la expresión como una
multiplicación entre ( b\_j ) y la suma de los ( a\_i )
correspondientes. Esto reduce el número de multiplicaciones al agrupar
las operaciones.

\subsubsection{Código en Python}\label{cuxf3digo-en-python}

A continuación, se presenta un código en Python para calcular ambas
sumas y compararlas:

\begin{Shaded}
\begin{Highlighting}[]

\KeywordTok{def}\NormalTok{ original\_sum(a, b, n):}
   
\NormalTok{    result }\OperatorTok{=} \DecValTok{0}
    \ControlFlowTok{for}\NormalTok{ i }\KeywordTok{in} \BuiltInTok{range}\NormalTok{(n):}
        \ControlFlowTok{for}\NormalTok{ j }\KeywordTok{in} \BuiltInTok{range}\NormalTok{(i }\OperatorTok{+} \DecValTok{1}\NormalTok{):}
\NormalTok{            result }\OperatorTok{+=}\NormalTok{ a[i] }\OperatorTok{*}\NormalTok{ b[j]}
    \ControlFlowTok{return}\NormalTok{ result}

\KeywordTok{def}\NormalTok{ reorganized\_sum(a, b, n):}
   
\NormalTok{    result }\OperatorTok{=} \DecValTok{0}
    \ControlFlowTok{for}\NormalTok{ j }\KeywordTok{in} \BuiltInTok{range}\NormalTok{(n):}
\NormalTok{        sum\_a }\OperatorTok{=} \BuiltInTok{sum}\NormalTok{(a[j:])  }\CommentTok{\# Suma de los a[i] desde i = j hasta n}
\NormalTok{        result }\OperatorTok{+=}\NormalTok{ b[j] }\OperatorTok{*}\NormalTok{ sum\_a}
    \ControlFlowTok{return}\NormalTok{ result}

\CommentTok{\# Ejemplo de uso}
\NormalTok{a }\OperatorTok{=}\NormalTok{ [}\DecValTok{1}\NormalTok{, }\DecValTok{2}\NormalTok{, }\DecValTok{3}\NormalTok{, }\DecValTok{4}\NormalTok{]  }\CommentTok{\# Valores de a\_i}
\NormalTok{b }\OperatorTok{=}\NormalTok{ [}\DecValTok{5}\NormalTok{, }\DecValTok{6}\NormalTok{, }\DecValTok{7}\NormalTok{, }\DecValTok{8}\NormalTok{]  }\CommentTok{\# Valores de b\_j}
\NormalTok{n }\OperatorTok{=} \BuiltInTok{len}\NormalTok{(a)  }\CommentTok{\# Longitud de las listas}

\CommentTok{\# Cálculo de ambas sumas}
\NormalTok{original }\OperatorTok{=}\NormalTok{ original\_sum(a, b, n)}
\NormalTok{reorganized }\OperatorTok{=}\NormalTok{ reorganized\_sum(a, b, n)}

\BuiltInTok{print}\NormalTok{(}\StringTok{"Suma original:"}\NormalTok{, original)}
\BuiltInTok{print}\NormalTok{(}\StringTok{"Suma reorganizada:"}\NormalTok{, reorganized)}
\end{Highlighting}
\end{Shaded}

\subsection{DISCUSIONES}\label{discusiones}

\subsubsection{2.}\label{section}

Las ecuaciones (1.2) y (1.3) en la sección 1.2 proporcionan formas
alternativas para las raíces (x\_1) y (x\_2) de {[} ax\^{}2 + bx + c =
0. {]} Construya un algoritmo con entrada (a, b, c) y salida (x\_1,
x\_2) que calcule las raíces (x\_1) y (x\_2) (que pueden ser iguales con
conjugados complejos) mediante la mejor fórmula para cada raíz.

\begin{Shaded}
\begin{Highlighting}[]
\ImportTok{import}\NormalTok{ math}

\KeywordTok{def}\NormalTok{ calcular\_raices(a, b, c):}
    

\NormalTok{    discriminante }\OperatorTok{=}\NormalTok{ b}\OperatorTok{**}\DecValTok{2} \OperatorTok{{-}} \DecValTok{4}\OperatorTok{*}\NormalTok{a}\OperatorTok{*}\NormalTok{c}

    \ControlFlowTok{if}\NormalTok{ discriminante }\OperatorTok{\textgreater{}=} \DecValTok{0}\NormalTok{:  }\CommentTok{\# Raíces reales}
        \CommentTok{\# Fórmula tradicional (generalmente más estable para raíces reales)}
\NormalTok{        x1 }\OperatorTok{=}\NormalTok{ (}\OperatorTok{{-}}\NormalTok{b }\OperatorTok{+}\NormalTok{ math.sqrt(discriminante)) }\OperatorTok{/}\NormalTok{ (}\DecValTok{2}\OperatorTok{*}\NormalTok{a)}
\NormalTok{        x2 }\OperatorTok{=}\NormalTok{ (}\OperatorTok{{-}}\NormalTok{b }\OperatorTok{{-}}\NormalTok{ math.sqrt(discriminante)) }\OperatorTok{/}\NormalTok{ (}\DecValTok{2}\OperatorTok{*}\NormalTok{a)}
    \ControlFlowTok{else}\NormalTok{:  }\CommentTok{\# Raíces complejas}
        \CommentTok{\# Fórmula tradicional (la parte imaginaria se calcula aparte para evitar problemas numéricos)}
\NormalTok{        parte\_real }\OperatorTok{=} \OperatorTok{{-}}\NormalTok{b }\OperatorTok{/}\NormalTok{ (}\DecValTok{2}\OperatorTok{*}\NormalTok{a)}
\NormalTok{        parte\_imaginaria }\OperatorTok{=}\NormalTok{ math.sqrt(}\OperatorTok{{-}}\NormalTok{discriminante) }\OperatorTok{/}\NormalTok{ (}\DecValTok{2}\OperatorTok{*}\NormalTok{a)}
\NormalTok{        x1 }\OperatorTok{=} \BuiltInTok{complex}\NormalTok{(parte\_real, parte\_imaginaria)}
\NormalTok{        x2 }\OperatorTok{=} \BuiltInTok{complex}\NormalTok{(parte\_real, }\OperatorTok{{-}}\NormalTok{parte\_imaginaria)}

    \ControlFlowTok{return}\NormalTok{ x1, x2}

\CommentTok{\#Complejos}
\NormalTok{a }\OperatorTok{=} \BuiltInTok{float}\NormalTok{(}\BuiltInTok{input}\NormalTok{(}\StringTok{"Ingrese el valor de a: "}\NormalTok{))}
\NormalTok{b }\OperatorTok{=} \BuiltInTok{float}\NormalTok{(}\BuiltInTok{input}\NormalTok{(}\StringTok{"Ingrese el valor de b: "}\NormalTok{))}
\NormalTok{c }\OperatorTok{=} \BuiltInTok{float}\NormalTok{(}\BuiltInTok{input}\NormalTok{(}\StringTok{"Ingrese el valor de c: "}\NormalTok{))}
\NormalTok{raices }\OperatorTok{=}\NormalTok{ calcular\_raices(a, b, c)}
\BuiltInTok{print}\NormalTok{(}\StringTok{"Las raíces son:"}\NormalTok{, raices)}
\end{Highlighting}
\end{Shaded}

\begin{verbatim}
Ingrese el valor de a:  4
Ingrese el valor de b:  5
Ingrese el valor de c:  6
\end{verbatim}

\begin{verbatim}
Las raíces son: ((-0.625+1.0532687216470449j), (-0.625-1.0532687216470449j))
\end{verbatim}

\begin{Shaded}
\begin{Highlighting}[]
\CommentTok{\#Reales}
\NormalTok{a }\OperatorTok{=} \BuiltInTok{float}\NormalTok{(}\BuiltInTok{input}\NormalTok{(}\StringTok{"Ingrese el valor de a: "}\NormalTok{))}
\NormalTok{b }\OperatorTok{=} \BuiltInTok{float}\NormalTok{(}\BuiltInTok{input}\NormalTok{(}\StringTok{"Ingrese el valor de b: "}\NormalTok{))}
\NormalTok{c }\OperatorTok{=} \BuiltInTok{float}\NormalTok{(}\BuiltInTok{input}\NormalTok{(}\StringTok{"Ingrese el valor de c: "}\NormalTok{))}
\NormalTok{raices }\OperatorTok{=}\NormalTok{ calcular\_raices(a, b, c)}
\BuiltInTok{print}\NormalTok{(}\StringTok{"Las raíces son:"}\NormalTok{, raices)}
\end{Highlighting}
\end{Shaded}

\begin{verbatim}
Ingrese el valor de a:  1
Ingrese el valor de b:  8
Ingrese el valor de c:  1
\end{verbatim}

\begin{verbatim}
Las raíces son: (-0.12701665379258298, -7.872983346207417)
\end{verbatim}




\end{document}
